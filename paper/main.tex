\documentclass[runningheads]{llncs}
\usepackage[english]{babel}
\usepackage[utf8x]{inputenc}
\usepackage[T1]{fontenc}

\usepackage{amsmath,amssymb}
%\usepackage[square,numbers,sort&compress]{natbib}
\usepackage{xspace}
%% Useful packages
\usepackage{graphicx}
\usepackage{pdflscape}
\usepackage{multirow}
\usepackage{booktabs} % For formal tables
\usepackage{enumitem}
\usepackage{afterpage}
\usepackage[colorlinks=true, allcolors=blue]{hyperref}
\usepackage[square,numbers,sort]{natbib}
% Fixes for using natbib and llncs
\makeatletter
% required for natbib to have "References" printed and as section*, not chapter*
\renewcommand\bibsection%
{\section*{\refname\@mkboth{\MakeUppercase{\refname}}{\MakeUppercase{\refname}}}}
\renewcommand\@biblabel[1]{#1.}
\def\bibfont{\small}
\makeatother

% If you use the hyperref package, please uncomment the following line
% to display URLs in blue roman font according to Springer's eBook style:
\renewcommand\UrlFont{\color{blue}\rmfamily}

\DeclareMathOperator*{\argmax}{arg\,max}

%% Revision documents
\usepackage[dvipsnames]{xcolor}
\usepackage[normalem]{ulem}
% FIXME: \sout is not very good. It cannot handle commands within its argument.
\newcommand{\SuggestEdit}[3][red]{\textcolor{#1}{\sout{#2}}\textcolor{#1}{#3}}
\let\svthefootnote\thefootnote
\newcommand\colorfootnote[2][black]{\def\thefootnote{\color{#1}\svthefootnote}%
  \footnote{\color{#1}#2}\def\thefootnote{\color{black}\svthefootnote}}
\newcommand\RevComment[3][red]{\protect\colorfootnote[#1]{{\textbf{[#2: #3]}}}}
% Command for edits, command for commenting and color
\newcommand\newrevisor[3]{%%
  \colorlet{#1}{#3}
  \expandafter\newcommand\csname#1\endcsname[2]{\SuggestEdit[#1]{##1}{##2}}%%
  \expandafter\newcommand\csname#2\endcsname[1]{\RevComment[#1]{#2}{##1}}%%
}
\newrevisor{manuel}{MANUEL}{Purple}

\hyphenation{%%% Merriam-Webster
  di-men-sion-al %%
  op-tical net-works semi-conduc-tor %%
  pher-o-mone non-dom-i-nance %%
  non-dom-i-nat-ed chro-mo-some %%
  sto-chas-tic make-span an-a-lys-ing}


\title{Mallows, Black-box combinatorial optimization, limited budget,  ???}
% Double-blind
\author{Double-blind}
% \author{Ekhiñe Irurozki\inst{1} \and Manuel López-Ibáñez\inst{2}\orcidID{0000-0001-9974-1295}}
% \institute{
%    Basque Center for Applied Mathematics\\
%    \email{eirurozki@bcamath.org}
%    \and
%    University of Málaga, Málaga, Spain\\
%    \email{manuel.lopez-ibanez@uma.es}
% }
\date{}%

\begin{document}

\maketitle

\begin{abstract}
%The abstract should briefly summarize the contents of the paper in
%150--250 words.
  Black-box combinatorial optimization problems arise in practice when the
  objective function is evaluated by means of a simulator or a real-world
  experiment. In such cases, classical techniques such as mixed-integer
  programming and local search cannot be applied. Moreover, often each solution
  evaluation is expensive in terms of time or resources, thus only a limited
  number of evaluations is possible, typically several order of magnitude
  smaller than in white-box optimization problems. In the continuous case,
  Bayesian optimization, in particular using Gaussian processes, has proven
  very effective under these conditions. Much less research is available in the
  combinatorial case. In this paper, we propose and analyze an
  estimation-of-distribution (EDA) algorithm based on a Mallows probabilistic
  model and compare it with CEGO, a Bayesian optimization algorithm for
  combinatorial optimization. Experimental results show that the Bayesian
  algorithm is able to obtain very good solutions with very few evaluations,
  however, at a significant computational cost, whereas the proposed EDA
  outperforms CEGO when the number of solutions evaluated approaches 400, and
  it is significantly faster. These results suggest that the combination of
  Bayesian optimization and a Mallows-based EDA may be an interesting direction
  for future research.
\textcolor{red}{16 pages max, deadline: 1 November 2020}
\keywords{Combinatorial optimization \and Bayesian optimization \and Expensive black-box optimization \and Estimation of distribution algorithms}
\end{abstract}

\section{Introduction}

Motivation: Manuel

In many practical optimization problems, the objective function is not
explicitly available and solutions are evaluated by means of expensive
prediction models, simulations or physical experiments. When decision variables
are continuous, the use of Bayesian surrogate-models, e.g., Gaussian processes,
in optimization has become
widespread~\citep{JonSchWel98go,ForKea2009surrogate}. Motivated by this
success, there have been attempts at adapting such Bayesian optimization
algorithms to the combinatorial case, a notable example being Combinatorial
Efficient Global Optimization
(CEGO)~\citep{ZaeStoBar2014:ppsn,ZaeStoFriFisNauBar2014}. However, the
ruggedness of combinatorial landscapes, which makes local search particularly
effective in the white-box context, lessens the effectiveness of global
surrogate models~\citep{EriPeaGar2019scalable}. Moreover, surrogate models are
expensive to train and optimize. Unless each function evaluation requires a day
or more, the time required by the Bayesian optimizer itself may impose a
significant overhead in computation time. \citet{PerLopStu2015si} recently
showed that ant colony optimization (ACO)~\citep{DorStu2004:book} is
competitive with CEGO on a black-box version of the travelling salesman problem
under a budget of $1\,000$ function evaluations. ACO basically builds a
probability distribution model from which solutions are sampled with a bias
towards the best solutions evaluated so far. As such, ACO may be considered a
type of estimation-of-distribution algorithm (EDA) for combinatorial
optimization.

In this work, we propose and analyze uMM, an EDA specifically designed for
black-box combinatorial optimization under a limited budget. The proposed uMM \ldots\MANUEL{what is the inspiration}
\

% minimize the uncertainty in the final solutions or

Related work: Ekhine

Our contribution: Ekhine
\begin{itemize}
\item 
\end{itemize}


\citep{LopDubPerStuBir2016irace}

\section{Background}

Permutations are defined as bijections of the set $[n]$ integer onto itself. The set of all permutations of $n$ items is denoted as $S_n$ and has cardinality $n!$. We denote permutations with Greek letters with exception of the inverse permutation denoted as $e=1, 2, 3, \ldots,n$. We denote the composition of $\sigma$ and $\pi$ as $\sigma\pi$ and the inverse of $\sigma$ as $\sigma^{-1}$, for which the relation $\sigma\sigma_{-1}=e$ always holds. 

Distributions over permutations are functions that assign a probability value to each of the permutations in $S_n$, $p(\sigma)\in[0,1]$ \cite{critchlow91}. One of the most popular distributions is the Mallows Model (MM), which is considered as an analogous to the Gaussian distribution for permutations. The MM defines the probability of each permutation $\sigma$ as follows:

\begin{equation}
p(\sigma)=\frac{\exp(-\theta d(\sigma, \sigma_0))}{\psi}
\end{equation}

with two parameters, $\theta$ and $\sigma_0$, 
where permutation $\sigma_0$ a reference permutation that has the largest probability value, i.e., the mode of the distribution. The probability of every permutation $\sigma\in S_n$ decays exponentially as its distance $d(\sigma,\sigma_0)$ increases, and $\theta$, the dispersion parameter controls this decay. The distance $d(\sigma,\sigma_0)$ is the Kendall's-$\tau$ distance. The normalization constant $\psi$ can be easily computed for the Kendall's-$\tau$ distance as well as for the Hamming, Cayley and Ulam distance~\cite{Irurozki2016b}. 

One of the most common problems related ti probability distributions is that of learning the maximum likelihood parameters  given a sample of data $S$. For the MM, this problem tranlates to learning $\theta$ and $\sigma_0$ that best describe a sample of permutations. 

The learning process is divide in two stages: first, we estimate the central permutation of the distribution, $\hat\sigma0$ and, second, compute the dispersion parameter, $\hat\theta$. 

The exact maximum likelihood estimation is computationally hard~\cite{Dwork:2001:RAM:371920.372165}. However, the approximate learning requires polynomilal computational time and is guarantied to obtain high quality parameters~\cite{Caragiannis2013,Coppersmith:2010}. 

This process of sample $S$ is as follows: first, compute $\hat\sigma0$ with the Borda count algorithm. Borda orders the items $[n]$ by their \textit{Borda score} increasingly, where the Borda score $B(i)$ is the average of each position $i$, $B(i) \propto \sum_{t\in S}  \sigma_t(i)$. Second, the computation of $\theta$ is casted as a numerical optimization problem~\cite{Irurozki2016b}. 

Recently, it has been proposed an extension of Borda by the name of uBorda. uBorda considers a sample of permutations $S$ along with a weight $w(\sigma)$ for each permutation $\sigma$~\cite{}. Intuitively it is equivalent to replicating the permutations in the sample proportionally to their weight. It has been used to learn a MM in an evolving preference context. In this paper, we propose to use uBorda where each permutation in the sample $\sigma\in S$ is weighted by its fitness function $w(\sigma)=f(\sigma)$.




LOP: Ekhine

Quizás luego: PFSP y QAP

\section{Methods}


\manuel{CEGO: Manuel.}{}

Combinatorial Efficient Global Optimization
(CEGO)~\citep{ZaeStoFriFisNauBar2014} is an extension of the well-known EGO
method~\citep{JonSchWel98go} to unconstrained black-box combinatorial
optimization problems. In EGO, Gaussian process models are used as a surrogate
of the landscape of the expensive original problem. An optimization method
searches for solutions in the surrogate model by optimizing the expected
improvement criterion, which balances the expected mean and variance of the
chosen solution. Once a solution is chosen, it is evaluated on the actual
objective function and the result is used to update the surrogate-model,
hopefully increasing its predictive power.

CEGO replaces the Euclidean distance measure, used by the surrogate model in
EGO, with a distance measure appropriate to combinatorial
landscapes~\citep{ZaeStoBar2014:ppsn}, such as the Kendall distance for
permutations~\citep{?}. In CEGO, the surrogate model is explored by a GA with
crossover and mutation operators appropriate for the particular combinatorial
problem. The original paper notes that coupling the GA with local search does
not improve the results significantly since the model is anyway an inexact
estimation of the original objective
function~\citep[p.~875]{ZaeStoFriFisNauBar2014}.

What else do we need to say?

\newcommand{\minit}{\ensuremath{m_\text{ini}}\xspace}

\newcommand{\FEmax}{\ensuremath{\textit{FE}_{\max}}}

cosas de intro : 
This new methodology relies on the assumption that the fitness landscatpe is correlated with a distribution over the same space. 
Previous paper have confirmed this point
For this case study, we show that this case for the Mallows model under the Kendall distance and 



In this section, we introduce our main contribution: We present a new methodology for the optimization of black box functions with a minimal number of queries. This methodology is a probabilistic, population-based algorithm. 

Essentially, the algorithm  considers a sample of permutations along with the fitness of each of them. It is an iterative process in which the number of iterations is fixed. 
Then, at each iteration, UMM (i) estimates a surrogate distribution of the fitness function and (ii) proposes a good fitted individual. 


\section{Learning stage}
In this section




learn distri: uMM with param $\rho$ and binary search rho

sample: peakig distri. Borda is the 

Quality guarantees

prrof, probem-distri correlation


\section{Experimental setup}

We use the implementation of GECO provided by the
authors\footnote{\url{https://cran.r-project.org/package=CEGO}}. Following the
authors of CEGO~\citep{ZaeStoFriFisNauBar2014,ZaeStoBar2014:ppsn}, we use and
the GA that optimizes the surrogate models uses a population size of 20,
crossover rate of 0.5, mutation rate of 1, tournament selection of size 2 and
probability of 0.9, interchange mutation (i.e., exchanging two randomly
selected elements of the permutation) and cycle crossover for permutations. The
budget of each run of the GA is \textcolor{red}{$10^4$} evaluations using the
surrogate-model. Although it is never stated in the original paper, the	`
implementation of CEGO generates a set of initial solutions of size
$\minit=10$ by means of a max-min-distance sequential design: new solutions
are added to the set sequentially by maximizing the minimum distance to
solutions already in the set. These initial solutions are then evaluated on the
actual objective function and the result is used to build the initial surrogate
model.% \citet{ZaeStoBar2014:ppsn} uses 10 

The CPU time required for a single run of CEGO using these settings is larger
than a week.  For simplicity, we use Kendall Tau distance~\citep{?} in all
experiments\MANUEL{\citet{ZaeStoBar2014:ppsn} claims this is called Swap
  distance, could you double-check?}. Nevertheless,
\citet{ZaeStoFriFisNauBar2014} points out that CEGO performs best on the QAP
when using Hamming distance. We plan to extend our analysis of uMM to other
distance measures and to the dynamic selection of distance
measures~\citep{ZaeStoBar2014:ppsn}.

In all experiments, we consider a maximum budget of $\FEmax=400$ evaluations of
the actual objective function. In a white-box context, state-of-the-art
algorithms for the problems considered here typically evaluate
\textcolor{red}{hundreds? thousands?} of solutions, thus, the budget considered
here for the black-box context is extremely limited.

We implemented UMM in Python 3. The parameter settings that we use are...

Complete this with details of uMM: MANUEL

\manuel{Computing system details: MANUEL}{%
  All experiments were run on Intel Xeon ``Ivybridge'' E5-2650v2 CPUs at 2.60\,GHz, 64\,GB RAM
  running CentOS Linux release 7.4.1708 (Core).  }


Describir como se generan las intancias de LOP: Ekhine

Describir las instancias: Ekhine

For the LOP, we generated synthetic instances of size $n=20$, $m=200$, $\phi\in\{0.5,0.7,0.9\}$\MANUEL{what is $m$?}

In the case of the QAP and PFSP, we consider the same instances as
\citet{ZaeStoFriFisNauBar2014,ZaeStoBar2014:ppsn}, i.e.,
\texttt{nug12},\texttt{nug30},\texttt{tho30} and \texttt{kra32} for the QAP,
with $n$ equal to the number in the instance name, and \texttt{reC05},
\texttt{reC13}, \texttt{reC19}, \texttt{reC31} for the PFSP, with
$n \in \{20, 20, 30, 50\}$, respectively.

\section{Experimental analysis}

Bayesian optimization methods using a global GP model, such as CEGO, are known
to have trouble optimizing locally \citep{EriPeaGar2019scalable}. Our
intuition is that this problem becomes worse in rugged combinatorial
landscapes, where small steps may produce drastic changes.

\section{Conclusions}

Open questions:
\begin{itemize}
\item How difficult is to extend uMM to other distance metrics?
\item Can we plot the posterior probability of the optimal solution?
\end{itemize}

\paragraph*{Acknowledgements.}

Thanks to Hao Wang (Leiden University) for pointing us to the arguments of
\citet{EriPeaGar2019scalable}.




\renewcommand{\doi}[1]{doi:\hspace{.16667em plus .08333em}\discretionary{}{}{}\href{https://doi.org/#1}{\urlstyle{rm}\nolinkurl{#1}}}
\bibliographystyle{splncs04nat}
\bibliography{optbib/abbrev,optbib/authors,optbib/journals,optbib/biblio,optbib/crossref}

\end{document}

%%% Local Variables:
%%% mode: latex
%%% TeX-master: t
%%% End:
